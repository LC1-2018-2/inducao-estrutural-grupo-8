\documentclass[a4paper, 10pt]{article}

\usepackage[utf8]{inputenc}
\usepackage[brazilian]{babel}

% The following packages can be found on http:\\www.ctan.org
\usepackage{graphics} % for pdf, bitmapped graphics files
\usepackage{epsfig} % for postscript graphics files
\usepackage{mathptmx} % assumes new font selection scheme installed
\usepackage{times} % assumes new font selection scheme installed
\usepackage{amsmath} % assumes amsmath package installed
\usepackage{amssymb}  % assumes amsmath package installed
\usepackage{proof} % para construir as arvores
\usepackage{listings} % para colocar código no latex

\title{\LARGE \bf
O Princípio da Indução e suas Aplicações
}

\author{Gabriel Alves 17/0033813, \\
        Henrique Mariano 17/0012280, \\
        Matheus Breder 17/0018997, \\
        Yuri Serka 17/0024385.}

\begin{document}
\maketitle

\begin{abstract}

Neste trabalho foi trabalhado os conceitos de Indução, que pode ser Fraca ou Forte e como elas se completam. Também foi proposto a análise da correção do algoritmo de ordenação $"Insertion Sort"$.

\end{abstract}

\section{Introdução}

A indução matemática é um poderoso artifício matemático utilizado para demonstrar a verdade de um número infinito de proposições. Há, no entanto, dois tipos de indução, a fraca e a forte.

A indução fraca toma como passo base mostrar que o enunciado vale para $n = b$ e então é feito o passo indutivo que consiste em mostrar que, se o enunciado vale para $n = k \geq{b} $, então o mesmo enunciado vale para $n = k + 1$.

Matemáticamente podemos escrever o Princípio da Indução Fraca, também conhecidada como Princípio da Indução Matemática, como:

$$
PIM = \begin{cases}
h_1 = P(0), &\mbox{onde } P\mbox{ é uma propriedade qualquer}\\
\forall{n}(P(n)), &\mbox{se}\quad \forall{m}(P(m) \implies P(m+1))
\end{cases}
$$

A indução forte ou princípio da boa ordem, toma como caso base mostrar que o enunciado vale para todos os números anteriores ao que se quer provar, se isso acontecer então a propriedade vale para este número.

Matemáticamente podemos escrever o Princípio da Indução Forte como:

$$
PIF = \forall{n}(P(n)), \quad\mbox{se}\quad\forall{k}(\forall{m}(m < k, P(m)) \implies P(k))
$$

Com isto podemos provar grande parte dos problemas de indução que forem propostos.

\section{Indução forte é equivalente a indução fraca}
\label{sec:inducao}

Queremos provar que a indução fraca complementa a indução forte e vice-versa, portanto temos algo como:

$$
    \forall{m}(P(m) \longrightarrow P(m+1)) \iff \forall{k}(\forall{m}(m < k, P(m)) \longrightarrow P(k))
$$

Portanto temos duas provas para analisar.

\begin{equation}
\label{pim2pif}
    \forall{m}(P(m) \longrightarrow P(m+1)) \implies \forall{k}(\forall{m}(m < k, P(m)) \longrightarrow P(k))
\end{equation}

Primeiro foi tentado resolver usando dedução natural como será mostrado a seguir.


$$
\infer[(\forall{i})] {\forall{k}(\forall{m}(m < k, \varphi[\frac{x}{m}])\longrightarrow\varphi[\frac{x}{k}])} {
    \infer[(\rightarrow_i), x] {\forall{m}(m < k, \varphi[\frac{x}{m}])\longrightarrow \varphi[\frac{x}{k_0}]} {
        \infer[(\forall{e})] {\varphi[\frac{x}{k_0}]} {
            \infer[(\forall{i})] {\forall{x}\varphi} {
                \infer[(\rightarrow_e)] {\varphi[\frac{x}{m_0+1}]} {
                    \infer[(\forall{e})] {m_0 < k, \varphi[\frac{x}{m_0}]} {
                        {[\forall{m}(m < k, \varphi[\frac{x}{m}])]^x}
                    }
                    & \infer[(\forall{e})] {\varphi[\frac{x}{m_0}] \longrightarrow \varphi[\frac{x}{m_0+1}]} {
                        \forall{m}(\varphi[\frac{x}{m}] \longrightarrow \varphi[\frac{x}{m+1}])
                    }
                }
            }
        }
    }
}
$$

No entanto ainda há dúvidas se esta prova está correta, pois já nos primeiros ramos da árvore temos: $m_0 < k, \varphi[\frac{x}{m_0}] \quad e \quad \varphi[\frac{x}{m_0}]\longrightarrow\varphi[\frac{x}{m_0+1}]$ e não temos total certeza se os dois $\varphi[\frac{x}{m_0}]$ são iguais.

Todavia, apenas analisando a equação (\ref{pim2pif}) e entendendo-a chega-se a conclusão de que, pela hipótese temos que se $\varphi[\frac{x}{k}] = T \implies \varphi[\frac{x}{k+1}] = T$, e como na indução forte temos de provar que a propriedade vale para todos os menores que um t qualquer, quando pararmos em t-1, então por hipótese teremos que $\varphi[\frac{x}{(t-1)+1}] = \varphi[\frac{x}{t}] = T$ que é exatamente o que queremos provar.

\begin{equation}
\label{pif2pim}
    \forall{k}(\forall{m}(m < k, P(m)) \longrightarrow P(k)) \implies \forall{m}(P(m) \longrightarrow P(m+1))
\end{equation}

Foi tentado montar a árvore para esta prova também. Abaixo está mostrado o que conseguimos fazer:

$$
\infer[(\forall{i})] {\forall{m}(\varphi[\frac{x}{m}]) \longrightarrow \varphi[\frac{x}{m+1}])} {
    \infer[(\rightarrow_i), x] {\varphi[\frac{x}{m_0}]) \longrightarrow \varphi[(\frac{x}{m_0+1}])} {
        \infer[(\forall{e})] {\varphi[\frac{x}{m_0+1}]} {
            \infer[(\forall{i})] {\forall{x}\varphi} {
                \infer[(\rightarrow_e)] {\varphi[\frac{x}{k_0}]} {
                    [\varphi[\frac{x}{m_0}]]^x
                    & \infer[(\forall{e})] {m_0 < k_0, \varphi[\frac{x}{m_0}]\rightarrow\varphi[\frac{x}{k_0}]} {
                        \infer[(\forall{e})] {\forall{m}(m < k_0, \varphi[\frac{x}{k_0}]\rightarrow\varphi[\frac{x}{k_0}])} {
                            \forall{k}(\forall{m}(m < k, \varphi[\frac{x}{m}])\longrightarrow\varphi[\frac{x}{k}])
                        }
                    }
                }
            }
        }
    }
}
$$

Analisando a equação (\ref{pif2pim}) temos que se a propriedade vale para todos os anteriores a k, ou seja, $\forall{m: m < k}, \varphi[\frac{x}{m}] = T$, e definindo o conjunto com todos os elementos menores a k como sendo $\mathbb{M}$, então se pegarmos qualquer i $\in \mathbb{M}$, teremos $i < k$. Mas e se irmos além e selecionar um elemento j, tal que j = i - 1, logo $j < i$, como tanto i como j são menores que k então são verdadeiros pela hipótese.
Agora temos que i pode ser provado por j, pois ao aplicarmos a indução fraca teremos: $\varphi[\frac{x}{j}] = T$ e $\varphi[\frac{x}{i}] = T$, mas $\varphi[\frac{x}{j}] = \varphi[\frac{x}{i-1}]$ então $\varphi[\frac{x}{i}]$ deve ser verdade, que é o que a indução forte garante.

\section{insertion sort é correto}
\label{sec:insert}

\section{Tentativas anteriores}

Podemos dizer que a prova que encontra-se abaixo, decorre dos esforços da equipe em outra tarefa: Comprovar a corretude do algoritmo merge\_sort. Desse modo, conseguimos trazer as experiências adquiridas com esta tarefa, para cá, e criar a prova que se segue. Já com o auxílio de uma definição de sorted? e outras, em um documento de teoria do pvs, presente em nosso repositório.  

\subsection{Algoritmo correto}

Um algoritmo é definido como correto se cumpre as responsabilidades a ele denominadas. Ou seja, por exemplo, um algortimo geral de ordenação é correto se ao ser aplicado sob qualquer lista de objetos reconhecidamente ordenáveis, retornar uma lista de mesma cardinalidade, e com os mesmo termos, ordenados. 

\subsection{Um teorema que afira a corretude}

Considerando os pseudo-códigos abaixo, que definem o algoritmo insertionSort, recursivamente, nós temos:
 
$$
InsertionSort(l) = \begin{cases}
l,  &\mbox{se } l\mbox{ é nula}\\
insert(h,insertionSort(l')), &\mbox{se } l = h::l'
\end{cases}
$$

Sendo o insert: 

$$
Insert(x, l) = \begin{cases}
x::null,  &\mbox{se } l\mbox{ é nula}\\
x::l, &\mbox{se } l = h::l' \mbox{  e  } x \leq h \\
h::(Insert(x,l')), &\mbox{se } l = h::l' \mbox{  e  }  x > h
\end{cases}
$$

Desse modo, para realizar uma prova correta acerca da corretude do algoritmo, podemos definir uma função que nos indique se há ou não uma lista ordenada (em linguagem do pvs): 

\begin{lstlisting}[tabsize = 4]
    sorted?(l:list[nat]) : RECURSIVE boolean =
	   CASES l OF
	   	 null: TRUE,
		 cons(h,tl): CASES tl OF
		 	     null: TRUE,
			     cons(hh,ttl): (h <= hh) AND sorted?(tl)
			     ENDCASES
	   ENDCASES
	   MEASURE length(l)
\end{lstlisting}

Basicamente esta função checa recursivamente, para cada termo da lista, a partir de sua cabeça, se este termo é menor ou igual ao termo posterior. Se isso ocorre em todos os casos, então temos uma lista ordenada. Também há o tratamento especial para o caso em que recebe-se uma lista nula: A mesma é considerada ordenada. Poderíamos então, definir em pseudo-código:

$$
sorted?(l) = \begin{cases}
TRUE,  \quad \mbox{se } l\mbox{ é nula}\\
h::l' = \quad \begin{cases} 
TRUE, &\mbox{se } l'\mbox{ é nula}\\
TRUE, &\mbox{se para } u::l'': (h \leq u) \quad AND \quad sorted?(l'')
\end{cases}\\
FALSE, \quad \mbox{caso contrário}
\end{cases}
$$

Certamente existem definições melhores para este algoritmo. No entanto, o mais importante aqui é considerar que sorted? nos indica se há uma lista ordenada, segundo os conceitos já explicitados mais acima. 

Desse modo, podemos construir o seguinte teorema, o qual iremos provar, e que ao ser provado nos levará ao resultado que esperamos: Provaremos que insertionSort é um algoritmo correto: 

$$ \forall{l: list[reais]}\quad sorted?(l) \implies sorted?(insertionSort(l)) $$

Este teorema nos diz que: Se houver uma lista não ordenada, então teremos uma lista ordenada após aplicar o InsertionSort, ou se houver uma lista ordenada, então ainda teremos uma lista ordenada ao aplicar o insertionSort. 

Mais adiante, para nos auxiliar, podemos considerar a seguinte função:

$$
antecedente(x, l) = \begin{cases}
b,  &\mbox{se } b::l'\quad  e \quad l' = x::l''\\
antecedente(x, l'), &\mbox{se } l = h::l' \quad e \quad l' = u::l'' \quad e \quad u \neq x \\
null, &\mbox{se } l' \mbox{ é nula} 
\end{cases}
$$

\subsection{Indução forte no teorema que afere a corretude}

Desse modo, podemos provar o nosso teorema por meio de uma indução forte, no conjunto dos reais. No entanto, essa prova também valeria para qualquer lista de objetos reconhecidamente ordenáveis: 

Comecemos criando dois passos base: 

 $$i)\quad l = null, \mbox{ onde l é uma lista} $$

Assim aplicando insertionSort em l temos uma lista ordenada por definição.

$$ii)\quad l = a, \mbox{ onde l é uma lista e } a \in \mathbb{R} $$ 

Assim aplicando insertionSort em l termos uma lista ordenada por definição. 

Podemos então criar o nosso passo indutivo no tamanho da lista (que é a característica que é reduzida ao passo da recursão):

$$iii) \forall{K, M: list[reais]}\quad e \quad length(K) < length(M): $$
É verdadeiro:
$$ sorted?(K)  \implies sorted?(insertionSort(K)) $$
Ou seja, insertionSort(K) é uma lista ordenada.
Sendo que:
$$ M = a::insertionSort(K), \mbox{ uma lista de reais} $$ 
então temos:
$$ insertionSort(K) = h::insertionSort(K)', \mbox{ h é a cabeça da lista.}$$
temos pela definição de insertionSort, quando aplicarmos insertionSort(M):
$$ Insert(a, insertionSort(K)) $$,
mas pela hípotese de indução iii), insertionSort(K) é uma lista ordenada. Então temos os casos:

$$1) \quad a \leq h $$ 
então:
$$ a::insertionSort(K) $$ é o retorno, que por definição é uma lista ordenada (a cabeça é menor ou igual que a cabeça de insertionSort(K), que também é uma lista ordenada). Ou seja: sorted?(a::insertionSort(K)) retorna TRUE. 

$$2) \quad a > h $$ 
então:
$$ h::insert(a, insertionSort(K)') $$ é o retorno. 

Neste caso podemos dizer que:

$$*)\quad \exists{j}, \mbox{ tal que } a \leq j,$$ 
ou
$$**)\quad \mbox{a encontra o termo nulo} $$ 

Logo, se temos *) então sabemos que:

$$ a > antecedente(insertionSort(K)', j) \quad a \leq j $$, portanto ao inserir a após o antecedente(insertionSort(K), j), ainda temos uma lista ordenada, pois insertionSort(K) é ordenado pela hipótese de indução iii). Ou seja sorted?(h::insert(a, insertionSort(K)')) retorna TRUE.

e se temos **) então sabemos que por definição a será inserido após o último termo não nulo de insertionSort(K), e:
$$ a > antecedente(insertionSort(K)', null) $$, ou seja, a é maior que o termo antecedente ao nulo. Logo, ainda temos uma lista ordenada. Ou seja sorted?(h::insert(a, insertionSort(K)')) retorna TRUE.

Assim, insertionSort sempre ordena uma lista sobre a qual é aplicado. 
$$C.Q.D$$


\section{Conclusão}
Vemos com este exercício a grande utilidade da prova por indução. Como também as grandes dificuldades apresentadas ao tentar utilizar a lógica matematica corretamente, por exemplo, ao tentar aplicar a dedução natural na seção \ref{sec:inducao}. 
De todo modo, como observado na seção \ref{sec:insert} a indução é muito poderosa em casos reais, em que queremos descobrir se um algoritmo é correto. Vemos também, que a necessidade de programas que auxiliam na construção de provas, como por exemplo, o pvs, é grande devido a maior velocidade para executar os pasos da prova, e em verificar a recursão dos algoritmos definidos aqui em pseudo-código. 
De qualquer forma, e com este trabalho, podemos ter um pouco do gosto do grande mundo que são as abstrações na computação: Estivemos passeando entre diferentes abstrações computacionais e matemáticas, na tentativa de comprovar que, uma máquina, ao executar um algoritmo, irá retornar sempre o resultado esperado aos olhos humanos. Essa é uma das utilidades das abstrações, mas sabemos que a própria ciência, ou mesmo a computação, e os avanços científicos, são construídos por meio de abstrações que modelam adequadamente a realidade. 
Deste modo, podemos dizer que foi um exercício de extrema valia para a equipe, os conhecimentos aqui praticados e esperamos que tenha sido uma ótima leitura, para o leitor. 

\end{document}
