\documentclass[a4paper, 10pt]{article}

\usepackage[utf8]{inputenc}
\usepackage[brazilian]{babel}

% The following packages can be found on http:\\www.ctan.org
\usepackage{graphics} % for pdf, bitmapped graphics files
\usepackage{epsfig} % for postscript graphics files
\usepackage{mathptmx} % assumes new font selection scheme installed
\usepackage{times} % assumes new font selection scheme installed
\usepackage{amsmath} % assumes amsmath package installed
\usepackage{amssymb}  % assumes amsmath package installed

\title{\LARGE \bf
O Princípio da Indução e suas Aplicações
}

\author{Gabriel Alves 17/0033813, \\
        Henrique Mariano 17/0012280, \\
        Matheus Breder 17/0018997, \\
        Yuri Serka 17/0024385.}

\begin{document}
\maketitle

\begin{abstract}

Neste trabalho foi trabalhado os conceitos de Indução, que pode ser Fraca ou Forte e como elas se completam. Também foi proposto a análise da correção do algoritmo de ordenação $"Insertion Sort"$.

\end{abstract}

\section{Introdução}

A indução matemática é um poderoso artifício matemático utilizado para demonstrar a verdade de um número infinito de proposições. Há, no entanto, dois tipos de indução, a fraca e a forte.

A indução fraca toma como passo base mostrar que o enunciado vale para $n = b$ e então é feito o passo indutivo que consiste em mostrar que, se o enunciado vale para $n = k \geq{b} $, então o mesmo enunciado vale para $n = k + 1$.

Matemáticamente podemos escrever o Princípio da Indução Fraca, também conhecidada como Princípio da Indução Matemática, como:

$$
PIM = \begin{cases}
h_1 = P(0), &\mbox{onde } P\mbox{ é uma propriedade qualquer}\\
\forall{n}(P(n)), &\mbox{se}\quad \forall{m}(P(m) \implies P(m+1))
\end{cases}
$$

A indução forte ou princípio da boa ordem, toma como caso base mostrar que o enunciado vale para todos os números anteriores ao que se quer provar, se isso acontecer então a propriedade vale para este número.

Matemáticamente podemos escrever o Princípio da Indução Forte como:

$$
PIF = \forall{n}(P(n)), \quad\mbox{se}\quad\forall{k}(\forall{m}(m < k, P(m)) \implies P(k))
$$

Com isto podemos provar grande parte dos problemas de indução que forem propostos.

% explicar melhor ainda

A indução estrutural é parecido com a indução forte, mas é comumente usada para provar conjuntos definidos recursivamente. A semelhança vem pelo fato de que o passo base é mostrar que o enunciado vale para todos os casos da definição da recursão, já o passo indutivo consiste em verificar se o enunciado vale para todos estes subconjuntos, então deve valer para o atual.

\section{Indução forte é equivalente a indução fraca}

\section{insertion sort é correto}

\subsection{Algoritmo correto}

Um algoritmo é definido como correto se cumpre as responsabilidades a ele denominadas. Ou seja, por exemplo, um algortimo geral de ordenação é correto se ao ser aplicado sob qualquer lista de naturais, retornar uma lista de mesma cardinalidade, e com os mesmo termos, ordenados. 

\subsection{Um teorema que afira a corretude}

\section{Conclusão}

\end{document}
